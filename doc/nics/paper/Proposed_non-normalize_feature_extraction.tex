%
% Project name   :
% File name      : Proposed_non-normalize_featture_extraction.tex
% Created date   : Mon 17 Jul 2017 10:39:51 AM +07
% Author         : Ngoc-Sinh Nguyen
% Last modified  : Mon 17 Jul 2017 10:39:51 AM +07
% Desc           :
%
\section{Proposed cell histogram generation}
\label{sec:proposed_non_normalize_feature_extraction}
In this work, we proposed a robust and low complex methodology to determine the
two  angle and calculate the magnitude of the two voted bins.
This methodology has no calculation of the gradient of pixel.
To determine the two quantized orientations exactly, a trusted form of Equation \ref{eq:angle_bound}
was discovered.
It is based on the limitation of the $\frac{d_y}{d_x}$ and  the $\tan\theta$ values.
Gradient magnitude of two bins are the solutions of a system
of two equations instead of voting gradient.
%gradient magnitude of two voted orientation are the roots of a system of two
%equations.

\subsection{Robust quantized angles determination}
\label{sub:robust_quantized_angles_detection}

Equation \ref{eq:theta} shows our basic ideas. 
In stead of choosing fixed bit length of fractional part of $\tan\theta$ or
representing $\tan\theta$ by structure $\sum{2^{i}}$, our proposed method
bounds each $\tan\theta$ by two rational numbers: its nearest smaller $\frac{d_y}{d_x}$, called
$\frac{A_s}{B_s}$ and its nearest greater  $\frac{d_y}{d_x}$, called $\frac{A_g}{B_g}$.
Those fractions did not choose from the \textit{tan} only, it was from examining all
cases of $\frac{d_y}{d_x}$ and \textit{tan} values to reach the expected results.
\begin{equation}
	\frac{A_s}{B_s} < \tan\theta < \frac{A_g}{B_g} \label{eq:theta}
\end{equation}
Table \ref{tab:tanphi_and_its_nearest_values} shows all $\frac{A_s}{B_s}$ and
$\frac{A_g}{B_g}$ in a case  of 8-bit length pixel. 
As shown, with those rational numbers, we do not need to multiply with very
large number such as $10^5$ to determine quantized orientation exactly.

\begin{table}[h]
	\centering
	\caption{$\tan\theta$ and its nearest values $\frac{d_y}{d_x}$ }
	\label{tab:tanphi_and_its_nearest_values}
	\begin{tabular}{|c|c|c|}
		\hline
		%\rowcolor{lightgray}
	\textbf{Smaller and Nearest $|\frac{d_y}{d_x}|$} & $\mathbf{\tan\theta}$ & \textbf{Greater and Nearest $|\frac{d_y}{d_x}|$ }
		\\\hline
		$\frac{43}{244}$ &	$\tan(10)$ & $\frac{3}{17}$
		\\\hline
		$\frac{56}{57}$  & $\tan(30)$  & $\frac{97}{168}$
		\\\hline
		$\frac{230}{193}$ & $\tan(50)$ & $\frac{87}{73}$
		\\\hline
		$\frac{250}{91}$ & $\tan(70)$ & $\frac{11}{4}$
		\\\hline
	\end{tabular}
\end{table}
%\begin{align}
Equation \ref{eq:theta} leads to equivalent equations Equation \ref{eq:theta_i} - \ref{eq:theta_i+1}
for determining two quantized orientations: $\theta_i$ and $\theta_{i+1}$, respectively.
As shown, there are only multiplication and comparison operations at this
moment, and it is much easier than inverse tangent for hardware implementation.


%\begin{subequations}
	\begin{IEEEeqnarray}{cc}
		\alpha &> \theta 
			\Leftrightarrow \left\{ \begin{array}{rcl}
				|\frac{d_y}{d_x}| > \frac{A_s}{B_s} \\
				|\frac{d_y}{d_x}| \ge \frac{A_g}{B_g}
			\end{array}\right.
			\Leftrightarrow \left\{ \begin{array}{rcl}
				B_s \times |d_y| > A_s \times |d_x| \\
				B_g \times |d_y| \ge A_g\times |d_x| \\
			\end{array}\right.  \label{eq:theta_i} \\
		\alpha &< \theta
			\Leftrightarrow \left\{ \begin{array}{rcl}
				|\frac{d_y}{d_x}| \le \frac{A_s}{B_s} \\
				|\frac{d_y}{d_x}| < \frac{A_g}{B_g}
			\end{array}\right.
			 \Leftrightarrow \left\{
			\begin{array}{rcl}
				B_s \times |d_y| \le A_s \times |d_x| \\
				B_g \times |d_y| < A_g\times |d_x| \\
			\end{array}\right.  \label{eq:theta_i+1}
	\end{IEEEeqnarray}
%\end{subequations}
where 
\begin{itemize}
	\item $\frac{A_s}{B_s}$ is a $\frac{d_y}{d_x}$ number, and it is the nearest smaller or equal number of $\tan\theta$ value;
	\item $\frac{A_g}{B_g}$ is a $\frac{d_y}{d_x}$ number, and it is the nearest greater or equal number of $\tan\theta$ value;
	\item $A_s, B_s, A_g, B_g$ is integer number in $|d_x|,|d_y|$ range.
	\item $\theta$ is quantized orientation;
	\item $\alpha$ is the gradient orientation of the current pixel.
\end{itemize}

\subsection{Magnitude of two bins calculation}
\label{sub:magnitude_of_two_bins_calculation}
As discussion in section
\ref{sub:determining_two_quantized_bins_without_gradient_orientation} results of
bin's boundary angles do not provide enough arguments for voting. 
Our work  proposed a system of two equations from information: $\theta_i,
\theta_{i+1}, d_x$ and $d_y$ to calculate gradient magnitude of two bins
exactly.

Back to the ideas of HOG, it decomposes gradient of a pixel $I(x,y)$ into
a combination of two bins as Figure \ref{fig:M_Oxy} and  Equation \ref{eq:m_to_b_b}.
Hence, in each direction, the sum  of two bins has to equal the
gradient of the pixel.
Equation \ref{eq:x_to_b_b} and \ref{eq:y_to_b_b} present those equalities
in the \textit{Ox} and \textit{Oy} axises, respectively.
Finally, the two magnitudes are the solutions of the two
equations, and  have the form  as Equation \ref{eq:b1} and
Equation \ref{eq:b2}, respectively.  

\begin{figure}[t]
	\def\sscale{1.0\linewidth}
	\includegraphics[width=\sscale]{"imgs/M_Oxy"}
	\caption{Decomposition a vector into form of two vectors}
	\label{fig:M_Oxy}
\end{figure}

\begin{IEEEeqnarray}{rl}
	\graM &= \graBi + \graBii  \label{eq:m_to_b_b}	 \\
	\magM \times \cos\alpha
		&=  d_x \nonumber \\
		&= \magBi \cos\theta_i + \magBii \cos\theta_{i+1} \label{eq:x_to_b_b}	 \\
	%
	\magM \times \sin\alpha
		&= d_y \nonumber \\
		&= \magBi \sin\theta_i + \magBii \sin\theta_{i+1} \label{eq:y_to_b_b}\\
	%
	\magBi  &= \frac{sin(\theta_{i+1})d_x - cos(\theta_{i+1})d_y }{sin(20)} \label{eq:b1} \\
	\magBii &= \frac{cos(\theta_i)d_y - sin(\theta_i)d_x}{sin(20)} \label{eq:b2}
\end{IEEEeqnarray}

The competitive advantages of this methodology, it  allows computing gradient
magnitude  with only $d_x$, $d_y$ information and the two quantized angles.
The most complex part are $\sin, \cos$ values, but they are
pre-computed.
By defining those numbers, the accurate level is manipulated completely.
To achieve these good, this method requires determining the quantized
orientations exactly.
If there is any errors in determining direction of two bins, it produces negative magnitudes.

\subsection{Proposed cell histogram generation scheme}
\label{sub:proposed_cell_histogram_generation_scheme}
Figure \ref{fig:dataflow}  shows the data flow of proposed cell histogram
generation, which employed our methodology.
As shown, the input contains 4 neighbor pixels $I(x+1)(y), I(x-1,y), I(x, y +
1), I(x, y - 1)$  of pixel $I(x,y)$.
The first step, the gradient in \textit{x} and \textit{y} directions  $d_x$ and $d_y$ are computed.
At the second step, we have absolute values of $d_x$ and $d_y$, and the
quadrant position of the gradient of pixel $I$.
If the sign of $d_x$ is opposite  sign of $d_y$, the gradient $\graM$ is in
the second quadrant.
Otherwise, it is in the first quadrant.
Because the left side of \textit{Oy} axis is a reflection of the right side of
\textit{Oy}, so computing the magnitude of two voted bins in the first quadrant
is enough.
The third step is to determine the two quantized angles $\theta_i, \theta_{i+1}$ via comparison
between $A \times d_x$ and $B \times d_y$, where $A$ is the dividend,
and $B$ is the divisor in Table
\ref{tab:tanphi_and_its_nearest_values}.
In this scheme, we use the nearest greater values
$\frac{d_y}{d_x}$ of $\tan\theta$.
Depend on situation, you are able to choose the left side of the Table
\ref{tab:tanphi_and_its_nearest_values}.
After having two quantized angles, the two magnitudes $\magBi, \magBii$ are from
Equations \ref{eq:b1} and \ref{eq:b2}.
Finally, we have to re-correct the angles from quadrant position information.
If those angles are in the second quadrant, converting  the $\theta$ in the first
quadrant into the second quadrant is done by a subtraction $\theta = 180 - \theta$.

\begin{figure}[h]
	\def\sscale{1.0\linewidth}
	\includegraphics[width=\sscale]{"imgs/dataflow"}
	\caption{Data flow of proposed methodology}
	\label{fig:dataflow}
\end{figure}

As shown in Figure~\ref{fig:dataflow}, all
of the multiplication operations, we have already known one of two numbers.
It allows converting all multiplications into form of shifts and additions.
Table \ref{tab:mult_2_shift_add} shows  all of the transformations
with 8-bit length of fractional part in \textit{sine} and \textit{cosine}.
It is equivalent to multiply 256.
The size of fractional part of magnitude is controlled by the \textit{sine}, \textit{cosine}
values in the Table \ref{tab:mult_2_shift_add}.
For example, if  an application needs 10-bit length of fractional part, the
\textit{cosine}, \textit{sine} values have to multiply 1024 instead of 256.
From Table \ref{tab:mult_2_shift_add} and 
Figure~\ref{fig:dataflow}, the total number of operations to identify two voted
bins are about 30 additions and nearly 40  shifts.

%  -----------------------------------------------------------------------------

\begin{table}[h]
	\centering
	\caption{Converting all multiply to shift and addition operations with 8-bit
	of fractional part}
	\label{tab:mult_2_shift_add}
	\begin{tabular}{|c|c|c|c|}
		\hline
		%\rowcolor{lightgray}
		\textbf{$B \times d_y$} & \textbf{Shift and addition} & \textbf{$A \times d_x$} & \textbf{Shift and addition}
		\\\hline
		4                       & $2^2$                       & 11                      & $2^3 + 2^1 + 2^0$
		\\\hline
		73                      & $2^6 + 2^3 + 2^0$           & 87                      & $2^6 + 2^4 + 2^3 - 2^0$
		\\\hline
		168                     & $2^7 + 2^5 + 2^3$           & 97                      & $2^6 + 2^5 + 2^0$
		\\\hline
		17                      & $2^4 + 2^0$                 & 3                       & $2^1  + 2^0$
		\\\hline
		$cos(10) \times 256$    & $2^8 - 2^2$                 & $sin(10) \times 256$   & $2^5 + 2^3 + 2^2$
		\\\hline
		$cos(30) \times 256$    & $2^8 - 2^5 - 2^1$           & $sin(30) \times 256$   & $2^7 $
		\\\hline
		$cos(50) \times 256$    & $2^7 + 2^5 + 2^2 + 2^0$     & $sin(50) \times 256$   & $2^7 + 2^6 + 2^2$
		\\\hline
		$cos(70) \times 256$    & $2^6 + 2^4 + 2^3 $          & $sin(70) \times 256$   & $2^8 - 2^4 + 2^0$
		\\\hline
		$cos(90) \times 256$    & $0$                         & $sin(90) \times 256$    & $2^8$
		\\\hline
		$\frac{1}{sin(20)}$     & $2  + \frac{15}{16}$        &                         &
		\\\hline
	\end{tabular}
\end{table}

