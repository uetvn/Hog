%
% Project name   :
% File name      : intro.tex
% Created date   : Mon 17 Jul 2017 10:42:25 AM +07
% Author         : Ngoc-Sinh Nguyen
% Last modified  : Mon 17 Jul 2017 10:42:25 AM +07
% Desc           :
%

\section{Introduction}
\label{sec:introduction}
The Histogram of Oriented Gradients (HOG) \cite{dalal2005hog} is a feature descriptor used in
computer vision and image processing for object detection.
It gets the quantity of strong characteristic from the shape change of the object by
dividing a local domain into plural blocks and making the incline of each the
histogram.
Practically, HOG feature achieves very high accuracy level,  it is up to
96.6\% of the detection rate with 20.7\% of the false positive rate
\cite{negi2011dpo}.
Hence, it is the key role of wide  range application domains including robotic,
security surveillance, automotive.
However,  the complexity level of HOG is the most difficult problem when
using in embedded systems or implementing into hardware.


The HOG algorithm can be separated into 3 phases: cell histogram generation,
block normalization, and the SVM classification.
The first phase plays as the most complex one and consumes the most energy.
Because, extracting the feature of each pixel requires series of  complex
operations: inverse tangent, square, square root, floating point multiplication.
Additionally, in the age of cloud computing, the two last phases can be done in the
server  for IoT devices.
Hence, simplifying cell histogram generation part is  very necessary,
especially for IoT devices.

Many researchers have proposed diverse approximate methods such as LUT or CORDIC
or bin's boundary angles in cell histogram generation to reduce complexity
level.
Even though approximate models reduce the  accuracy of extracting pixel feature,
they still require too much memory areas or use many iterations.

%
In this work, we  would like to propose a new methodology, which extracts
features of pixel accurately and is suitable for low resource systems and
hardware implementation.
The difference in this methodology is that it is not an approximate model, and it
computes the quantized orientations and gradient magnitude of quantized organized directly.
The idea comes from the properties of decomposing the gradient of a pixel into
two components (bins).
The proposed methodology includes only addition and shift operation.
As our simulation results, it requires about 30 additions
and 40 shifts for 8-bit of fraction of magnitude.
In the accurate side, error when reconstructing the pixel's derivatives in both
direction from voted bins is always less than 2\%.
A hardware implementation  consumes about 3.57\textit{KGates} with 45\textit{nm}
NanGate standard cell library.
The maximum frequency of the hardware is 400\textit{MHz}, and its throughput is up to 0.4\textit{pixel/ns}.


The rests of the paper are organized as follows.
Section \ref{sec:conventional_non_normalize_feature_extraction_in_hog}
introduces the cell histogram generation and some previous optimization works.
Section \ref{sec:proposed_non_normalize_feature_extraction} shows the detail of
proposed methodology.
Section \ref{sec:experimental_results} presents details of the hardware
implementation of proposed methodology and simulation results.
Finally, section \ref{sec:conclusion} gives a summary and our expected
future.
