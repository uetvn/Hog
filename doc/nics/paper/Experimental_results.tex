%
% Project name   :
% File name      : Experimental_results.tex
% Created date   : Mon 17 Jul 2017 11:16:25 AM +07
% Author         : Ngoc-Sinh Nguyen
% Last modified  : Mon 17 Jul 2017 11:16:25 AM +07
% Desc           :
%


\section{Hardware implementation and experimental results}
\label{sec:experimental_results}
%\subsection{Hardware implementation}
%\label{sub:hardware_pipeline_model}

We implemented the proposed methodology into hardware and examined
its results.
Figure \ref{fig:hw_implementation_002} shows the hardware module of
the proposed methodology.
It includes 4 input pixels $I(x + 1, y), I(x - 1, y), I(x, y + 1), I(x, y - 1)$
with 8-bit pixel length.
The output contains the magnitude and the direction of two bins.
Angle  is 4-bit length data, and magnitude is 17-bit length data with
8-bit of fractional part.

\begin{figure}[h]
	\centering
	\def\sscale{1\linewidth}
	\includegraphics[width=\sscale]{"imgs/hw_implementation"}
	\caption{Hardware implementation of proposed methodology}
	\label{fig:hw_implementation_002}
\end{figure}



As shown in Figure~\ref{fig:hw_implementation_002}, the module includes 4
states.
Firstly, the difference $d_x$ and $d_y$ are computed from the 4 input
pixels.
In the second state, the absolute of $d_x$, $d_y$ are computed, and the quadrant
position of the gradient of pixel $I(x,y)$ is  detected.
After having those absolute values, the pre-calculating state calculates
all the multiplications in Table \ref{tab:mult_2_shift_add}.
The fourth state produces two voted bins, include  orientation and magnitude.

Figure \ref{fig:angle_and_magnitude_calculation} presents the detail of the
fourth state.
It has two multiplexers.
The first multiplexer has 4 input from 4 angle comparisons.
This multiplexer chooses the highest angle $\theta$ in the first quadrant. 
For each quantized angle, it also produces corresponding gradient
magnitude from Equation \ref{eq:b1} and \ref{eq:b2}.
The second multiplexer converts the $\theta$ in first quadrant into the second
quadrant if the $sign(d_x)$ is opposite $sign(d_y)$.

\begin{figure}[h]
	\centering
	\def\sscale{.8\linewidth}
	\includegraphics[width=\sscale]{"imgs/angle_and_magnitude_calculation"}
	\caption{Submodule angle and magnitude calculation}
	\label{fig:angle_and_magnitude_calculation}
\end{figure}

Table \ref{tab:cell_histogram_generation_hardware_comparison} shows the hardware
results of some state-of-the-art publications and our proposed.
Pei-Yin and Hsiao's works calculated the gradient of  current pixels with
approximate model  via the SRA approximation for magnitude and CORDIC for angle.
Munteanu combined Equation \ref{eq:munteanu} and bin's boundary angles method.
As shown, the proposed methodology and Munteanu's works  can run at highest
frequency 400\textit{MHz}.
They are  also the two smallest areas cost of implementation.
However, the area of  Munteanu's module still more than   fifth times as much as
our proposed.


% Please add the following required packages to your document preamble:
% \usepackage{multirow}
% \usepackage[table,xcdraw]{xcolor}
% If you use beamer only pass "xcolor=table" option, i.e. \documentclass[xcolor=table]{beamer}
\begin{table}[h]
\centering
\caption{Hardware comparison of Cell Histogram Generation }
\label{tab:cell_histogram_generation_hardware_comparison}
\begin{tabular}{|l|c|c|c|}
\hline
%\rowcolor{lightgray}
\textbf{Publications} & \textbf{Technology} & \textbf{Areas (KGates)} & \textbf{Frequency} \\ \hline
Pei-Yin \cite{pei-yinchen2014ehia}          & 130\textit{nm}               & 85.5                    & 167\textit{MHz} (all HOG)   \\ \hline
Hsiao \cite{hsiao2016hdh}                   & 90\textit{nm}                & 29.1                    & N/A                \\ \hline
Munteanu \cite{munteanu2016mph}             & N/A                 & 20                      & Max. 400\textit{MHz}        \\ \hline
Our Proposed                                & 45\textit{nm} NanGate        & 3.57                    & Max. 400\textit{MHz}        \\ \hline
\end{tabular}
\end{table}



%\subsection{Verification model}
%\label{sub:experimental_results}

Figure \ref{fig:verification_model} shows our error of calculation model for the
proposed methodology.
From the voted bin, we calculate a model of $d_x$ and $d_y$ by Equation
\ref{eq:x_to_b_b} and \ref{eq:y_to_b_b}, called $d'_x$ and $d'_y$.
With the $d'_x$, $d'_y$, $d_x$, $d_y$, the percentage error of calculation at
both \textit{Ox} and \textit{Oy} axises are  computed.
Tested with all cases, the results prove that our proposed methodology provides
the  percentage error of calculations $e_x$ $e_y$, which  are always  less than 2\% with 8-bit length of
fractional part.
\begin{figure}[h]
	\centering
	\def\sscale{0.8\linewidth}
	\includegraphics[width=\sscale]{"imgs/verification_model"}
	\caption{Error of calculation model}
	\label{fig:verification_model}
\end{figure}


