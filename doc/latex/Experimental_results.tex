%
% Project name   :
% File name      : Experimental_results.tex
% Created date   : Mon 17 Jul 2017 11:16:25 AM +07
% Author         : Ngoc-Sinh Nguyen
% Last modified  : Mon 17 Jul 2017 11:16:25 AM +07
% Desc           :
%


\section{Implementing proposed methodology into hardware and experimental results}
\label{sec:experimental_results}
\subsection{Hardware implementation}
\label{sub:hardware_pipeline_model}

In this section, we implement the proposed methodology into hardware and examine
the results.
Figure \ref{fig:hw_implementation_002} shows the hardware module of
the proposed methodology.
It includes 4 input pixels $I(x + 1, y), I(x - 1, y), I(x, y + 1), I(x, y - 1)$
with 8-bits pixel length.
The output contains the 2 couples of angle and magnitude, which represents the
voted bins of current pixel $I(x,y)$.
The angle output is 4-bits length, and the magnitude is 17-bits length with
8-bits of fractional part.

\begin{figure*}[t]
	\centering
	\def\sscale{0.7\linewidth}
	\includegraphics[width=\sscale]{"imgs/hw_implementation"}
	\caption{Hardware implementation of proposed methodology}
	\label{fig:hw_implementation_002}
\end{figure*}
\begin{figure*}[t]
	\centering
	\def\sscale{.7\linewidth}
	\includegraphics[width=\sscale]{"imgs/verification_model"}
	\caption{Error of calculation model}
	\label{fig:verification_model}
\end{figure*}


As shown in Figure~\ref{fig:hw_implementation_002}, the module includes 4
states.
 Firstly, the difference $d_x$ and $d_y$ are computed from the 4 input
pixels.
In the second state, the absolute of $d_x$, $d_y$ are computed, and the quadrant
position of the gradient of the pixel $I(x,y)$ is  detected.
After having those absolute values, the pre-calculating state calculates
parallel all the multiples in Table \ref{tab:mult_2_shift_add}.
At the fourth state, the two voted angles are detected by only comparison
operations and one subtraction, and the magnitude is  from one subtraction as and
only 1 multiply with $\frac{1}{\sin(20)}$ as the Equation \ref{eq:b1} or
\ref{eq:b2}.

\subsection{Experimental results}
\label{sub:experimental_results}

Figure \ref{fig:verification_model} shows an error of calculation model of the
proposed methodology.
From the voted bin, we calculate a model of $d_x$ and $d_y$ by Equation
\ref{eq:x_to_b_b} \ref{eq:y_to_b_b}: $d'_x$ and $d'_y$.
With the $d'_x$, $d'_y$, $d_x$, $d_y$, the percentage error of calculation at both Ox $e_x$
and Oy $e_y$ axises are  computed.
Tested with all cases, the results prove that our proposed methodology provides
The  percentage error of calculations $e_x$ $e_y$ are always  less than 2\% with 8-bits length of
fractional part. Table \ref{tab:gate_count} compares the area cost of the
proposed design with a recent HOG hardware implementation in the
paper \cite{pei-yinchen2014ehia}, \cite{hsiao2016hdh} and in the patent
\cite{munteanu2016mph}, the area cost of cell histogram generation reduced to
3.5 Kgates with 45nm NanGate standard cell library.

% Please add the following required packages to your document preamble:
% \usepackage{multirow}
\begin{table}[]
\centering
\caption{Gate counts comparison of the different HOG hardware designs}
\label{tab:gate_count}
\begin{tabular}{|c|c|c|c|}
\hline
\multirow{2}{*}{} & \multirow{2}{*}{\begin{tabular}[c]{@{}c@{}}CMOS\\ Technology\end{tabular}} & \multicolumn{2}{c|}{Gate Count(Kgates)} \\ \cline{3-4}
                  &                                                                            & Gradient \& Magnitude  & Gradient vote  \\ \hline
HOG \cite{pei-yinchen2014ehia} & 130nm                                                                      & 0.5                    & 87.0           \\ \hline
HOG \cite{hsiao2016hdh}        & 90nm                                                                       & 1.3                    & 27.9           \\ \hline
Patent \cite{munteanu2016mph}  & N/A                                                                        & \multicolumn{2}{c|}{20.0}               \\ \hline
Proposed HOG      & 45nm                                                                       & \multicolumn{2}{c|}{3.6}                \\ \hline
\end{tabular}
\end{table}
