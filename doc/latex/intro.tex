%
% Project name   :
% File name      : intro.tex
% Created date   : Mon 17 Jul 2017 10:42:25 AM +07
% Author         : Ngoc-Sinh Nguyen
% Last modified  : Mon 17 Jul 2017 10:42:25 AM +07
% Desc           :
%

\section{Introduction}
\label{sec:introduction}
The histogram of oriented gradients (HOG) \cite{dalal2005hog} is a feature descriptor used in
computer vision and image processing for objection detection.
It gets the quantity of strong characteristic from the shape change of the object by
dividing a local domain into plural blocks and making the incline of each the
histogram.
Practically, HOG feature achieves very high accuracy level,  it is up to
96.6\% of the detection rate with 20.7\% of the false positive rate
\cite{negi2011dpo}.
Hence, it is the key role of wide  range application domains including robotic,
security surveillance, etc.
However,  the complexity level of HOG is the most difficult problem when
implementing it into embedded systems and  battery systems.


The HOG algorithm can be separated into 3 phases: cell histogram generation,
block normalization, and the SVM classification.
The first phase plays as the most complex one and consumes the most energy.
Because, extracting the feature of each pixel requires series of highly complex
operation: inverse tangent, square root, float multiply.
Moreover, in the age of cloud computing, the two last phases can be done in the
server  for IoT devices.
It means that the local devices only need to do cell histogram generation part,
then transferring all voted bins to their servers.
It allows extending the life of battery systems,

In this work, we study about the cell histogram generation and would like to optimize this
phase to be able to run in low resource embedded systems.
Our methodology  is transforming all of the complex operations inverse tangent, square root
and float multiply  into series of shift and addition operations.
As our experiments, voting gradient of a pixel into the two standard bins  requires about 30 addition operations
and 40 shift operations.
About the accuracy of computing, the percentage errors are only about 1\% of
$d_x$ $d_y$.
Hardware module of proposed methodology consumes about 3.57 KGates with 45nm
NanGate standard cell library.
The maximum frequency of the hardware is 400MHz, and its throughput is up to 0.4 pixel/ns.


The rests of the paper are organized as follows.
Section \ref{sec:conventional_non_normalize_feature_extraction_in_hog}
introduces the cell histogram generation and some previous optimization works.
Section \ref{sec:proposed_non_normalize_feature_extraction} shows the detail of
proposed methodology.
Section \ref{sec:experimental_results} presents details of the hardware
implementation of proposed methodology and simulation results.
Finally, section \ref{sec:conclusion} gives a summary and our expected
future.
