%
% Project name   :
% File name      : intro.tex
% Created date   : Mon 17 Jul 2017 10:42:25 AM +07
% Author         : Ngoc-Sinh Nguyen
% Last modified  : Mon 17 Jul 2017 10:42:25 AM +07
% Desc           :
%

\section{Introduction}
\label{sec:introduction}
However,  the complexity level of HOG is the most difficult problem when
implementing it into embedded systems and  battery systems.


The HOG algorithm can be separated into 3 phases: cell histogram generation,
block normalization, and the SVM classification.
The first phase plays as the most complexity one, and consumes the most energy.
Because, extracting feature of each pixel requires series of high complex
operation: $\arctan$, square root, float multiply.
Moreover, in the age of cloud computing, the two last phases can be done in the
server  for IoT devices.
It means that the local devices only need to do cell histogram generation part,
then transferring all voted bins to the servers.
It allows extending the life of battery systems,

In this work, we study about the cell histogram generation and would like optimize this
phase to be able to run in low resource embedded systems.
Our methodology  is transforming all complex operations $\arctan$, square root
and float multiply  into series of shift and addition operator.
As our experiments, voting to the two standard bins of pixels requires about 30 addition operations
and 40 shift operations.
About the accuracy of computing, the percentage errors are only about 1\% of
$d_x$ $d_y$.
\fixme{hw results}


The rests of the paper are organized as follows.
Section \ref{sec:conventional_non_normalize_feature_extraction_in_hog}
introduces the cell histogram generation and some previous optimization works.
Section \ref{sec:proposed_non_normalize_feature_extraction} shows the detail of
propose methodology.
Section \ref{sec:experimental_results} presents details of hardware
implementation of propose methodology and simulation results.
Finally, section \ref{sec:conclusion} gives summary and our expectation
future.
